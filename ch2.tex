\section{Reproducible Builds for Open-Source Software}
\nomenclature{FUSE}{Filesystem in Userspace}

\subsection[Reproducible Builds for Open-Source Software]{Motivation and history}
\nopagebreak[4]{
With the increasing popularity of open-source software \autocite{Anthes16a} and its adoption in large
scale systems, the security and quality assurance of that software becomes a
high-priority question. Commercial closed-source software projects rely on
internal quality assurance methods, and users of their products are expected
to trust results of such assurance. While it is a working model for many
companies, it may lead to cases when known or deliberately introduced flaws
in software are hidden from end-user, the most known example being 2015
Volkswagen emissions scandal \autocite{Sch15}. \\\\
FOSS, on the other hand, uses completely different
approach. While there is often no person directly responsible of quality assurance,
publicly available source code means that every interested user or company using
that software, may examine the code for potential flaws, therefore assuring
quality for themselves \autocite{hoepman2007increased}.
That model assumes that the source code ultimately defines the software product.
However, in reality people usually use software products in forms of binaries,
compared libraries or other form of build artifacts, which are the results of software build
process. In fact, it is more common for users to obtain software products in
form of build artifacts distributed by someone who built it, rather than
acquiring the source code and building it for themselves. \\\\
That consideration brings forth certain questions. Is source code really the only
thing that defines the output of software building process? Do other factors,
such as configuration of the build system, affect it? How can we be sure these
factors do not cause the modification of the resulting binary? If they do
modify it, then just ensuring the source code does not have any flaws does
not guarantee the resulting binary does not have them, since they could be introduced,
whether intentionally or not, by build system. \\\\
Open-source projects that place an extra emphasis on security
and privacy are especially concerned with this opportunity, as
it may leave the systems they are building vulnerable to cyberattacks.
It should come as no surprise that the Tor project was one of the
first to ring the alarm, alerting others of the problem and taking first steps
in ensuring what soon would be named reproducible builds \autocite{tor13, tor14}.
Since then, a lot of free software projects acknowledged the problem and
tried to come up with a solution. Individual projects concerned with quality
assurance of software they are producing started researching possible sources of
non-determinism in build process as well as ways to remove them, as can be
seen on example of TrueCrypt \autocite{de2014challenges}.\\\\
An important role in the following actions taken to raise awareness of the
problem was played by Debian, one of the most popular Linux distributions.
Over the last two years, members of Debian community gathered these efforts
under the flag of the project called Reproducible Builds \autocite{rb},
gave numerous talks on technical conferences on the matter
\autocite{Lun14, lca2017_valerie}, and organized
two summits dedicated to problem of reproducible builds. They also set up a continuous
testing infrastructure \autocite{tests-rbo}, which checks reproducibility
of Debian packages, and designed several tools helping to find
differences between build artifacts and identify common issues leading to
unreproducible builds.}
\subsection[Definitions]{Definitions}
\nopagebreak[4]{
As the aftermath the Reproducible Builds World Summit II (December, 2016),
representatives of projects taking part in the Reproducible Builds effort
agreed on following definition of reproducible build \autocite{rb-def}:
\theoremstyle{definition}
\begin{definition}{Reproducible build.} \\
A build is reproducible if given the same source code, build environment and build
instructions, any party can recreate bit-by-bit identical copies of all specified artifacts.
\\
\\
The relevant attributes of the build environment, the build instructions and the source
code as well as the expected reproducible artifacts are defined by the authors or
distributors. The artifacts of a build are the parts of the build results that are the
desired primary output.
\end{definition}


The following terms are used in this definition \autocite{rb-def}:
\begin{itemize}
    \item Source code is usually a checkout from version control at a
    specific revision or a source code archive.

    \item Relevant attributes of the build environment would usually include
    dependencies and their versions, build configuration flags and environment
    variables as far as they are used by the build system (e.g. the locale).
    It is preferable to reduce this set of attributes.

    \item Artifacts would include executables, distribution packages or
    filesystem images. They would not usually include build logs or similar ancillary
    outputs.

    \item The reproducibility of artifacts is verified by bit-by-bit comparison.
    This is usually performed using cryptographically secure hash functions.

    \item Authors or distributors means parties that claim reproducibility of
    a set of artifacts. These may be upstream authors, distribution maintainers or any
    other distributor.
\end{itemize}
}

\subsection[Tools]{Tools}
\nopagebreak[4]{

A number of tools was designed by the Reproducible Builds community to help open-source
projects check their products for reproducibility, find out what gets changed between
two builds and identify common unreproducibility issues.
\begin{itemize}
    \item Diffoscope is a tool for in-depth comparison of files, archives,
          and directories.
    \item Trydiffoscope is an online version of diffoscope.
    \item Disorderfs is an overlay FUSE (Filesystem in Userspace) filesystem that           
          deliberately introduces
          non-determinism into filesystem metadata.
    \item Strip-nondeterminism is a library for stripping non-deterministic
          information, such as timestamps and file system order, from files.
    \item Reprotest is a tool for building the same source code under
          different environments to check if these changes led to changes
          in the resulting binaries.
\end{itemize}

The Reproducible Builds project members have also come up with a
\texttt{SOURCE\_DATE\_EPOCH} specification \autocite{SDEspec},
defining a distribution-agnostic standard for build systems to exchange a
timestamp, which has proven to be one of the most common
source of reproducibility issues.
}




%\cleardoublepage
