{\centering
%
\SuperFont\MakeTextUppercase{Аннотация}\\
}
\vspace{\stretch{1}}
Данная работа ставит своей целью сделать краткий обзор понятия повторяемых сборок в сфере свободного программного обеспечения (ПО), шагов, предпринимаемых сообществом свободного ПО для их обеспечения и используемых для этого инструментах. В частности, более детально рассмотрен инструмент для сравнения результатов сборки -- diffoscope. В ходе работы было внесено несколько функциональных изменений в diffoscope с целью повышения удобства его использования для обеспечения репродуцируемых сборок.\\\\
Одним из главных достоинств свободного ПО является возможность изучать и модифицировать исходный код. Эта особенность также является гарантом безопасности и качества свободного ПО, так как любые недостатки или намеренные уязвимости в коде программы могут быть обнаружены любым её пользователем. Однако, чаще всего пользователи свободного ПО получают его уже в собранном виде. Чаще всего такое ПО распространяется в виде бинарных пакетов, доступных в репозиториях операционной системы либо на сайте производителя ПО. Сборка такого пакета при этом зачастую производится на специально выделенных для этого компьютерах либо на компьютерах разработчиков ПО.\\\\
Наличие дополнительного этапа, неподконтрольного пользователю, между исходным кодом и конечным исполняемым файлом или библиотекой, открывает новый фронт для кибератак и место для потенциальных уязвимостей. Эти уязвимости могут выражаться в том, что какие-либо особенности в процессе сборки могут влиять на результат неочевидным способом, внося, возможно, не предусмотренные разработчиком изменения в работу программы.\\\\
Эту опасность можно было бы предотвратить, повторяя процесс сборки на различных компьютерах и сравнивая результат. Если у кого-то из <<сборщиков>> он получился отличным от остальных, это повод заподозрить какие-либо изменения в конфигурации сборки, возможно, вызванные кибератакой на компьютер, на котором она совершалась. Для того, чтобы такой способ работал, требуется гарантия того, что в нормальных условиях сборка ПО на разных компьютерах с разной конфигурацией из одного и того же исходного кода всегда давала один и тот же результат. Такое свойство ПО носит название повторяемых, или репродуцируемых сборок (reproducible builds).\\\\
Повторяемость сборки ПО является свойством этого ПО и находится в области ответственности его автора. Однако, так как в рамках работы мы говорим о свободном ПО, обеспечение повторяемых сборок различного ПО является интересом всего сообщества свободного ПО в целом. В частности, эта проблема важна для дистрибутивов свободных операционных систем (ОС), так как многие из них предоставляют собранные пакеты пользователям и поэтому компьютеры, на которых сборка этих пакетов происходит, являются очевидной целью для возможных кибератак. По этой причине представители различных проектов по созданию свободного ПО создали проект, призванный повышать видимость проблемы, а также обсуждать и предлагать конкретные шаги и инструменты для проверки и обеспечения повторяемых сборок для свободного ПО.\\\\
Общий алгоритм проверки ПО на репродуцируемость прост: достаточно произвести сборку два раза, внеся некоторые изменения в систему, из которой производится сборка. Такими изменениями могут быть язык ОС, дата и время, текущая директория, имя пользователя и др. Предполагается, что подобные факторы не должны влиять на процесс сборки. Результаты сборок должны быть идентичны побитово; обычно это проверяется сравнением их хэш-сумм. Если же результаты не совпадают, необходимо выяснить, какие именно отличия есть между собранными артефактами. Это нужно для определения настоящей причины неповторяемости сборки.\\\\
Для поиска различий между двумя файлами участниками проекта Reproducible Builds был создан инструмент diffoscope. Этот инструмент написан на Python 3 и распространяется под свободной лицензией GNU General Public License третьей или выше версии. Он имеет интерфейс командной строки, хотя существует и веб-версия. Его основная задача -- ``раскрывать'' файлы и рекурсивно сравнивать их содержимое. Он распаковывает архивы, дизасемблирует некоторые объектные файлы, конвертирует PDF-файлы в текст и изображения и применяет другие различные инструменты и приёмы для перевода файла в текстовый вид, удобный для восприятия человеком.
