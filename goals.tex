\section{diffoscope improvement requests}

\nomenclature{APK}{Android Package}
\nomenclature{ZIP}{ZIP archive file format}
\nomenclature{JPEG}{Joint Photographic Experts Group - method of lossy compression for digital images}
\nomenclature{ICO}{Image file format for computer icons in Microsoft Windows}
\nomenclature{PNG}{Portable Network Graphics - raster graphics file format that supports lossless data compression}
\nomenclature{TAR}{(t)ape (ar)chive - computer software utility for collecting many files into one archive file}
\nomenclature{BZIP2}{free and open-source file compression program that uses the Burrows–Wheeler algorithm}
\nomenclature{BTS}{Bug tracking system}

As the diffoscope is used by different projects, some of them may have the unique needs, e.g. for some specific file type comparison.
Some other features, on the other hand, would improve overall functionality and usability of diffoscope and would therefore benefit all its users.\\\\
Object-oriented structure of the code makes it relatively easy to add support for
new file formats.
However, even for the file formats that are already supported, there is still work
to be done in order to make output more informative and to detect differences that
otherwise would be hidden.
Bugs and feature requests are usually submitted and reviewed at the Debian bug tracking system \autocite{dfs-bugs}.
In this work, several feature requests were fulfilled, mostly with focus on improved support for specific types of files.\\\\


\subsection[Order-like difference]{Order-like difference}
\nopagebreak[4]{
   Sometimes it can be useful to know if the two texts differ only in line
   numbering. While these differences should still be reported, as
   they mean the build is not reproducible, there should be a comment
   telling the inputs vary only in line ordering.
}

\subsection[APK files]{APK files}
\nopagebreak[4]{
   APK (Android Package) is the package file format used by the Android operating
   system for distribution and installation of mobile apps.
   It is essentially an ZIP-archive containing compiled source code, resources
   used by the application, and special control files.\\
   To retrieve human-readable information from APK files, external tool
   called \texttt{apktool} is used.\\
   Feature requests for APK files comparison in diffoscope include:
   \begin{itemize}[noitemsep,nolistsep]
    \item Handle APK metadata generated by apktool correctly.\\
    apktool, aside from other files, generates \texttt{apktool.yml}.
    That file contains meta information about package, such as Android SDK
    version and compression type. It also contains name of the input APK file.\\
    A request was made (bug \#850501 on Debian bug tracking system (BTS) \autocite{dfs-bugs})
    to use more appropriate name, identifying that information as APK metadata and
    not a part of package itself. It is also desirable that input filename be
    removed from this information, as the goal of comparison is compare file
    contents, not names.
    \item Add ZIP archive information.\\
    As APK files is just a specific type of ZIP archives, it is possible to
    extract ZIP archive specific information, such as access permissions,
    encryption status and type of compression, from them. A request was
    made (bug \#850502 on Debian BTS \autocite{dfs-bugs}) to
    include comparison of that kind of information in diffoscope output
    for APK files.
    \item Handle \texttt{AndroidManifest.xml} properly.\\
    \texttt{AndroidManifest.xml} file is an obligatory part of every Android package.
    The manifest file provides information about the application to the
    Android system, which the system must have before it can run any code of the
    application \autocite{andr}.\\
    Currently, \texttt{AndroidManifest.xml} is included in diffoscope output twice:
    its undecoded version, recovered by apktool, in XML format
    and its original (decoded) version in binary format. It may be better
    to show only undecoded version in most cases, as binary version is
    not informative when differences were already reported in more
    human-readable format.
    This idea comes from the feature request \#850758 on Debian BTS \autocite{dfs-bugs}
    \end{itemize}
}
\subsection[Image files]{Image files}
\nopagebreak[4]{
   Diffoscope has support for four types of image files: JPEG, MS Windows
   Icon (*.ico), GIF and PNG. Among these, JPEG and ICO images are compared
   using img2text tool, converting image into ASCII-graphics,
   allowing for easy comparison. PNG files are handled by \texttt{svg}
   tool, which extracts both metadata and image content (in text format), and GIF images are handled by \texttt{gifbuild}.\\\\
   There are two main feature requests for image comparison:
   \begin{itemize}[noitemsep,nolistsep]
   \item It would be useful to add metadata comparison for JPEG and ICO images.
   That would allow to detect differences in e.g. compression type, channel mode,
   color profile used, transparency setting.
   \item When HTML output option is used, it could be possible to
   display difference between image content in image format and not in
   pseudo-graphics.
   \end{itemize}
}
\subsection[Cross-container comparison]{Cross-container comparison}
\nopagebreak[4]{
   Unpacking archives to compare their contents and get to the bottom of
   what makes two files different is the main feature of diffoscope. It
   can process archives and containers of many kinds. However, it still
   expect both compared containers to have the same type.\\
   Some users would prefer being able to compare container content
   even when their type differs. For example, if one version
   of package is packed into ZIP archive and the other into TAR container
   with BZIP2 compression applied, it would be preferable to still compare
   their content. Right now diffoscope does not have this feature --
   it would treat these two packages as completely different files and
   compare them as binary files (byte-for-byte comparison).\\
}

%\cleardoublepage
