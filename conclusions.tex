\section{Conclusions}
\nopagebreak[4]{
  Reproducibility of software builds is important problem, critical for
  ensuring quality and security of open-source software. With the software
  building reproducibly, users can easily ensure no flaws were added to
  the product during build process and that the software indeed matches its
  source code. This problem attracted attention in various open-source projects,
  but outside of open-source world it is still not discussed enough. \\\\
  In this report, the history and motivation behind the
  Reproducible Builds project was presented. The definition of reproducibility
  in the software building process was given. An overview on the current reproducibility status of Debian operating system, as well as the steps taken to improve it, was discussed. Specifically discussed were the tools that the
  Reproducible Build project uses for testing reproducibility of software and
  identifying sources of unreproducibility. The particular tool for comparing
  two build outputs, named diffoscope, was discussed in detail, with emphasis on
  what can be done to improve it. The improvements for handling APK files, image files, containers and files with ordering-only difference were implemented and discussed as part of the work.\\\\
  While the significant progress has been made in spreading the word about the reproducibility problem, testing various software and fixing key reproducibility issues, the work is far from being done. With the new variations introduced in the testing process and new projects joining the initiative, it is mandatory that the work on categorizing the issues and fixing them continues as well. To make this process easier for all the interested parties, the tools used in the process should be constantly evolving to match the arising needs.\\\\
  The constant effort of people from different FOSS projects working on reproducible builds shows how highly the security and quality are valued in the world of open-source and how much is being done every day to keep the software we use secure.
}

%\cleardoublepage
