\section{Diffoscope}

This paper focuses on one of the tools being developed by
Reproducible Build project - diffoscope.
Diffoscope is a tool for in-depth comparison of files, archives and
directories\cite{dfs}.
The motivation behind development of diffoscope was to have a tool
telling what exactly have changed between two builds by comparing
their results. Therefore, main focus areas for this tool are:
\begin{itemize}
    \item Support for file formats that are used in build artifacts. \\
    Currently diffoscope supports various kinds of archives, packages,
    as well as some resources file formats, such as JSON data or PNG images.
    New file formats get constantly added by community.
    \item User-friendly output.\\
    Since the main goal of diffoscope is to provide users with a way to look
    into what makes two files different, its developers always try to
    improve the readability of output by hiding non-informative details and
    emphasizing the real differences.
\end{itemize}

Diffoscope is free software licensed under the GNU General Public
License version 3 or later. It is written in Python3 programming language.
It is available in the form of Debian package (unstable distribution),
Python package (PYPI), Homebrew package, Arch Linux package. Source code
of diffoscope is also available from Git repository at \cite{dfs-git}.
There is also an online version of diffoscope \cite{try-dfs}, so users
can try this tool without installing it on their system.
Bugs and feature requests can be submitted and reviewed at Debian
bug tracking system \cite{dfs-bugs}.\\
Community of diffoscope developers can be describing as having
onion model \cite{aberdour2007achieving}.
The project is being developed constantly by number of dedicated
developers and welcoming contributions as well as bug reports
and feature requests from everyone.
Diffoscope development follows the Open Source Development Model
\cite{osdm}, with frequent small releases and constant quality improvements.
These improvements are usually not planned in advance and often
done as result of resolving a bug or fulfilling a feature request.\\
In this paper, several improvements to diffoscope were implemented,
with focus on better support of various file formats and usability.
These improvements, too, were mostly dictated by bug and feature requests
received for diffoscope.



%\cleardoublepage
