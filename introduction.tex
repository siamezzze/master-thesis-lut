\chapter{Introduction}
\section[Background]{Background}
\setlength{\parskip}{1.0pt}
\nopagebreak[4]{
\nomenclature{FOSS}{Free and Open Source Software}
\nomenclature{OS}{Operating System}
One of the main ideas behind Free and Open Source Software
(FOSS) is that
anyone who uses the software can access the source code to make sure it does not perform 
any suspicious actions. However, in practice software
products are often used in form of build artifacts, which are results of software build
process. These artifacts are usually not human-readable, rather consisting of
binaries intended to run on the user machine or compiled libraries.
Thus, they cannot be inspected to ensure quality and security of the software. \\

The described approach introduces a potential flaw in quality assurance, where the result of
the build may be different from expected due to some changes on the build stage, either intended or accidental. That flaw can be exploited in order
to break software security, which is something that most open source software projects take very
seriously. Possibility of changes being introduced by build process
also have a negative impact on development process, since it becomes
impossible to tell for sure what changes have been introduced by the
changes in the source code, and which were caused by build process.
To solve this problem, people who develop and distribute software must ensure that their build process is deterministic with respect to different build environments.  \\

The Reproducible Builds project is a collective effort of multiple free software projects concerned
about quality and security of their products. The goal of the project is
to ensure that the software built under different environments
produces exactly the same results. If that goal is achieved, people who build the software would be able to compare results between each other,
and if someone's results are different, that would be a warning message meaning
their build system could be compromised.\\

To help software developers achieve reproducibility of their products,
several tools were created by the Reproducible Builds team.
These tools are designed for testing reproducibility of software, comparing build
results and identifying common issues.\\

This study describes how the reproducibility issues are identified, categorized and dealt with by Reproducible Builds project, and the tools used in this process. 
In particular, this thesis describes diffoscope, a tool that performs in-depth comparison for many file formats used in software development. This tool is widely used to identify reproducibility issues affecting software.
As part of this work, several features were added to diffoscope, enhancing its performance on specific types of files.
}

\setlength{\parskip}{1.0pt}
\section[Objectives and delimitations]{Objectives and delimitations}
\nopagebreak[4]{
The practical goal of this thesis is improvement of the diffoscope tool, making
it more useful for the Reproducible Builds project.
This work focuses on improvements made for specific types of files, aiming to provide a more human-readable comparison results for them. 
Diffoscope would also benefit from speed improvements,
but in this field the main task is making diffoscope run in parallel.
The difficulty of this task renders it too big for the scope of the
presented work.\\\\
Apart from the practical goal described above, this work aims
to raise awareness of the build reproducibility problem and ways to solve it. For this reason, a substantial part of
the report is dedicated to the details of the Reproducible Builds project, its motivation, goals,
proposed methods and achieved results.
}

\section[Structure of the thesis]{Structure of the thesis}
\nopagebreak[4]{
Section 2 explains in detail what the Reproducible Builds project is. Short
background of the problem and motivation for this collective effort are given.
This section also includes definition of reproducible builds, and gives a short
review of tools and solutions that the project has achieved so far.
Section 3 gives more details about diffoscope, the tool developed
as part of the Reproducible Builds effort. It
provides general information about the tool, its development model and
ways of distribution.
Sections 4 and 5 describe the process of adding new features to diffoscope.
Section 6 gives the overview of the current state of the Reproducible Builds project with focus on how the described tools are used in various situations.
Section 7 summarises the work and gives a short overview on what could be the possible next steps to address the discussed problem.

}



%\cleardoublepage
