\section{Conclusions}
\nopagebreak[4]{
  Reproducibility of software builds is important problem, critical for
  ensuring quality and security of open-source software. With the software
  building reproducibly, users can easily ensure no flaws were added to
  the product during build process and that the software indeed matches its
  source code. This problem attracted attention in various open-source projects,
  but outside of open-source world it is still not discussed enough. \\\\
  In this report, the history and motivation behind the
  Reproducible Builds project was presented. The definition of reproducibility
  in the software building process was given. An overview on the current reproducibility status of Debian operating system, as well as the steps taken to improve it, was discussed. Specifically discussed were the tools that the
  Reproducible Build project uses for testing reproducibility of software and
  identifying sources of unreproducibility. The particular tool for comparing
  two build outputs, named diffoscope, was discussed in detail, with emphasis on
  what can be done to improve it. \\\\
  The next step in the proposed work is to make some improvements to diffoscope: improve support
  of several file types and give user more control of the output.
}

%\cleardoublepage
