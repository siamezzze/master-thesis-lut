\section{Testing}

\subsection[Testing system of diffoscope]{Testing system of diffoscope}
\nopagebreak[4]{
Since the main body of diffoscope source code is written in Python, PyTest unit tests are used to 
ensure the new features work as intended and do not break the existing ones.
Unit tests cover the behavior of every part in diffoscope, such as comparators for the
different kinds of files, various types of output formats, aspects of user-defined 
behavior and more.
Any significant change in the diffoscope should be backed up with a few tests 
that make sure the new behavior matches the expected one. \\
As for now, diffoscope test suite contains around 300 unit tests.
}
\subsection[Compatibility with older versions]{Compatibility with older versions}
\nopagebreak[4]{
Diffoscope relies on many external tools to do its work. Some of them being crucial 
for the whole program to work while others can enhance its work in some cases.
These tools are considered to be coming from the user's operating system. In case of 
Debian, that means we are relying on the Debian repositories to provide user with 
necessary packages.\\
At the moment, diffoscope is available in Debian unstable and Debian stable backports 
distributions. Stable Debian distribution is older than unstable, and it has older versions
of required tools. To make sure stable version is properly supported, diffoscope should be tested
in both environments.\\
There was several issues related to backward compatibility that occurred and were resolved 
as part of the described work.


}
\subsection[New test cases]{New test cases}
\nopagebreak[4]{
  For every change introduced, there were several tests added.
  In particular, APK files comparison test suite was added to the project,
  and image comparison tests were refilled with tests checking image metadata extraction
  and comparison.
}


%\cleardoublepage
