\section{Introduction}
\subsection[Background]{Background}
\setlength{\parskip}{1.0pt}
\nopagebreak[4]{
\nomenclature{FOSS}{Free and Open Source Software}
\nomenclature{OS}{Operating System}
One of the main ideas behind Free and Open Source Software
(FOSS) is that
anyone who uses that software can access the source code to make sure the
software does not perform any suspicious action. However, in practice software
products are often used in form of build artifacts, which are results of software build
process. These artifacts are usually not human-readable, rather consisting of
binaries intended to run on the user machine or compiled libraries.
Thus, they cannot be inspected to ensure quality. \\

That approach introduces a potential flaw in quality assurance, where the result of
the build may be different from expected due to some changes on the building
system, either intended or accidental. That flaw can be exploited in order
to break software security, which is something that most open source software take very
seriously. Possibility of changes being introduced by build process
also have a negative impact on development process, since it becomes
impossible to tell for sure what changes have been introduced by
changes in the source code, and which were caused by build process.\\

The Reproducible Builds project aims at mitigating that flaw by comparing build
results from multiple parties and ensuring they all got the same results.
It is a collective effort of multiple free software projects concerned
about quality and security of their product. The goal of the project is
to ensure that open source software built under different environments
produces exactly the same results. If that goal is achieved, users
of the software would be able to compare build results between each other,
and if someone's results are different, that would be a warning message meaning
their build system could be compromised.\\

To help software developers achieve reproducibility of their products,
several tools were created by the Reproducible Builds team.
These tools are designed for testing reproducibility of software, comparing build
results and identifying common issues.\\

This report covers a process of improving diffoscope, a tool that
performs in-depth comparison for many file formats used in software development.
It is used widely by projects taking part in the Reproducible Builds.
In this work, several features are added to diffoscope, making it more
useful for reproducibility testing.
}

\setlength{\parskip}{1.0pt}
\subsection[Objectives and delimitations]{Objectives and delimitations}
\nopagebreak[4]{
The main goal of this work is improvement of the diffoscope tool, making
it more useful for the Reproducible Builds project.
This work focuses on improvements made for better file format support and
user interaction. Diffoscope would also benefit from being more time-efficient,
but in this field the main task is making diffoscope run in parallel.
The difficulty of this task renders it too big for the scope of the
presented work.\\\\
Another important part of open source software development is testing.
While it is hardly ever expressed as additional task, the development workflow
itself dictates that every major change in the project should automatically come
with tests for affected functionality. Although such tests are going to be added
for the improvements made as part of this work, the report does not discuss them in
details.\\\\
Apart from a practical goal of improving diffoscope, this work aims
to raise awareness of the build reproducibility problem and the collective
Reproducible Builds effort. For this reason, a substantial part of
the report is dedicated to the details of this project, its motivation, goals,
proposed methods and achieved results.
}

\subsection[Structure of the thesis]{Structure of the thesis}
\nopagebreak[4]{
Section 2 explains in detail what the Reproducible Builds project is. Short
background of the problem and motivation for this collective effort are given.
This section also includes definition of reproducible builds, and gives a short
review of tools and solutions that the project has achieved so far.
Section 3 gives more details about the specific tool developed
as part of the Reproducible Builds effort, diffoscope. It
provides general information about the project, its development model and
ways of distribution.
Section 4 focuses on specific problems this work is aiming to solve.
While the main goal is improvement of diffoscope according
to the Reproducible Builds project needs, some specific ways of
improvement are described in more details.
}



%\cleardoublepage
