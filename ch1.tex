\section{Introduction}
\subsection[Background]{Background}
\nopagebreak[4]{
\nomenclature{FOSS}{Free and Open Source Software}
\nomenclature{OS}{Operating System}
One of the main ideas behind Free and Open Source Software 
(FOSS) is that 
anyone who uses that software can access the source code to make sure the 
software does not perform any suspicious action. However, in practice software 
products are often used in form of build artifacts - results of software build 
process. These artifacts are usually not human-readable, rather consisting of 
binaries intended to run on the user machine or compiled libraries. 
Thus, they cannot be inspected to ensure quality. \\

That approach introduces a potential flaw in quality assurance, where the result of 
the build may be different from expected due to some changes on the building 
system, either intended or accidental. That flaw can be exploited in order 
to break software security - something that most open source software take very 
seriously. Possibility of changes being introduced by build process
also have a negative impact on development process, since it becomes
impossible to tell for sure what changes have been introduced by
changes in the source code, and which were caused by build process.\\

\textbf{Reproducible Builds} project aims at neglecting that flaw by comparing build 
results from multiple parties and ensuring they all got the same results.
It is a collective effort of multiple free software projects concerning
about quality and security of their product. The goal of the project is
to ensure that open source software built under different environments
produces exactly the same results. If that goal is achieved, users 
of the software would be able to compare build results between each other,
and if someone's results are different, that would be a warning message meaning 
their build system could be compromised.\\

To help software developers achieve reproducibility of their products, 
several tools were created by \textbf{Reproducible Builds} team.
These tools are designed for testing reproducibility of software, comparing build 
results and identifying common issues.\\

This paper covers a process of improving \textbf{diffoscope} - a tool that 
perform in-depth comparison for many file formats used in software development. 
It is used widely by projects taking part in \textbf{Reproducible Builds} effort, 
both independently and as part of the testing system. In this work, several 
features are added to \textbf{diffoscope}, making it more useful for 
reproducibility testing.
}

\subsection[Objections and delimitations]{Objections and delimitations}
\nopagebreak[4]{
Main goal of this work is improvement of \textbf{diffoscope} tool, making
it more useful for the \textbf{Reproducible Builds}.
Nature of open source development process makes it difficult to pin down
more specific objectives, as the requirements tend to shift all the time.
However, there are general focus points for improvements, which could be defined
as follows:
\begin{itemize}
    \item Improving support for individual file types.\\
    For some file formats, additional information could be gathered and compared;
    For some, on the other hand, it would be useful to limit an amount of information.
    This work also includes reacting to the changes made in external tools used
    to extract information from various kind of files and updated standards.
    \item Improving overall usability of \textbf{diffoscope}.\\
    There are some common issues not related to the specific file types,
    but rather to an overall readability of output and ease of use of the tool.
    In general, the goal here is to reduce the amount of noise (non-informative
    output) and detect the most significant differences between files.
    \item Improving run speed of \textbf{diffoscope}.\\
    In real applications, \textbf{diffoscope} is often run on big files
    like operating system (OS) images. It is used in continuous testing,
    where there would be a lot of packages to test. Therefore, time and memory
    resources are limited, and the tool cannot be allowed to run for too long.
    It is important to make \textbf{diffoscope} as time-efficient as possible
    without reducing its functionality.\\
\end{itemize}
This paper focuses on improvements made for the first two goals. While 
time-efficiency is very important, the main task in this field is, at the moment, 
making \textbf{diffoscope} run in parallel. The difficulty of this task renders it
too big for the scope of the presented work.\\
Another important part of open source software development is testing.
While it is hardly ever expressed as additional task, the development workflow
itself dictates that every major change in the project should automatically come 
with tests for affected functionality. Although such tests are going to be added
for the improvements made as part of this work, the paper will not discuss it in 
details.\\
Apart from a practical goal of improving \textbf{diffoscope}, this work aims
to raise awareness of the build reproducibility problem and the collective
\textbf{Reproducible Builds} effort. For this reason, a substantial part of
the paper is dedicated to the details of this project, its motivation, goals,
proposed methods and achieved results.\\
Results of the work are going to be discussed in the light of what does it
provide for the \textbf{Reproducible Builds} project enthusiasts and how does
it help to achieve the project's goals.
}

\subsection[Structure of the thesis]{Structure of the thesis}
\nopagebreak[4]{
Section 2 explain in detail what \textbf{Reproducible Builds} project is. Short 
background of the problem and motivation for this collective effort are given.
That section also include definition of reproducible builds, and gives a short 
review of tools and solutions that the project has achieved so far.\\
Section 3 gives more details about the specific tool developed
as part of \textbf{Reproducible Builds} effort - \textbf{diffoscope}. It 
provides general information about project, its development model and 
ways of distributing.\\
Section 4 focuses on specific problems this work is aiming to solve.
While the main goal is improvement of \textbf{diffoscope} according
to the \textbf{Reproducible Builds} project needs, some specific ways of
improvement are described in more details.
}



%\cleardoublepage
