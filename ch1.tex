\section{Introduction}

\nopagebreak[4]{
One of the main ideas behind Free and Open Source Software (FOSS) is that 
anyone who uses this software can access the source code to make sure the 
software does not perform any suspicious action. However, in practice software 
products are often used in form of build artifacts - results of software build 
process. These artifacts are usually not human-readable, rather consisting of 
binaries intended to run on the user machine. Thus, they cannot be
inspected to ensure quality. \\

That approach introduces a potential flaw in security, where the result of 
the build may be different from expected due to some changes on the building 
system, either intended or accidental.\\

Reproducible Builds effort aims at neglecting that flaw by comparing build 
results from multiple parties and ensuring they all got the same results. 
To make this process easier and to help software developers achieve 
reproducibility of their products, several tools are being designed. \\

This paper covers a process of improving \textbf{diffoscope} - a tool that 
perform in-depth comparison for many file formats used in software development. 
It is used widely by projects taking part in Reproducible Builds effort. \\
In this work, number of features are added to diffoscope, making it more 
useful for reproducibility testing.

%In order to understand the cause of heterogeneity in granular
%media, stress fluctuations are measured globally and locally
%\citep{Liu95}. Some measurements showed that the spatial
%distribution of force is highly inhomogeneous, which
%intermittently changes in any position with time \citep{Mil96}.
%Some models have been developed which could satisfactorily match
%the statistics of stress fluctuations in granular material
%\citep{De91,Liu95,Edw03,Gol04,Edw05}.\\}



\cleardoublepage
