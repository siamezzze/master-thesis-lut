

% макросы для начала введения и заключения
\newcommand{\Intro}{\addsec{Введение}}
\ifthenelse{\value{worktype} > 1}{%
    \renewcommand{\Intro}{\addchap{Введение}}%
}

\newcommand{\Conc}{\addsec{Заключение}}
\ifthenelse{\value{worktype} > 1}{%
    \renewcommand{\Conc}{\addchap{Заключение}}%
}

% Правильные значки для нестрогих неравенств и пустого множества
\renewcommand {\le} {\leqslant}
\renewcommand {\ge} {\geqslant}
\renewcommand {\emptyset} {\varnothing}

% N ажурное: натуральные числа
\newcommand {\N} {\ensuremath{\mathbb N}}

% значок С++ — используйте команду \cpp
\newcommand{\cpp}{%
C\nolinebreak\hspace{-.05em}%
\raisebox{.2ex}{+}\nolinebreak\hspace{-.10em}%
\raisebox{.2ex}{+}%
}

% Неразрывный дефис, который допускает перенос внутри слов,
% типа жёлто-синий: нужно писать жёлто"/синий.
\makeatletter
    \defineshorthand[russian]{"/}{\mbox{-}\bbl@allowhyphens}
\makeatother


%% Legacy imports/macros from Fin version.

%\usepackage[utf8]{inputenc}
%\usepackage[dvips]{graphicx} %[dvips] to used with .eps figures
%\usepackage{natbib}
%\usepackage{amsmath}
%\usepackage{amssymb}
%\usepackage{amsthm}
%\usepackage[main=english, finnish]{babel}
%\usepackage{csquotes}
%\usepackage[
%    backend=biber,
%    style=numeric,
%    citestyle=numeric 
%]{biblatex}
%\addbibresource{ref.bib}
\usepackage[notintoc]{nomencl} % including table of content
%\usepackage{ifthen}
%\usepackage[hmargin=3.0cm,vmargin=3.6cm]{geometry} % setting marginals
%\usepackage{fancyhdr}  % header ja footer manipulation
%\usepackage{times}  % to change font to times
%\usepackage{setspace} % for linespacing
\usepackage{url}
\usepackage{hyperref}
\usepackage{enumitem}
\usepackage[figure,table]{totalcount}
\usepackage[section]{placeins}
\hypersetup{
    colorlinks,
    citecolor=black,
    filecolor=black,
    linkcolor=black,
    urlcolor=black
}

\newcommand{\nomunit}[1]{\renewcommand{\nomentryend}{\hspace*{\fill}#1}} % Inserts units on the right at symbol list
\renewcommand{\nomgroup}[1]{%
 \ifthenelse{\equal{#1}{C}}{\item[\textbf{Latin alphabet}]\item}{%
 \ifthenelse{\equal{#1}{G}}{\item\item[\textbf{Greek alphabet}]\item}{}}{%
 \ifthenelse{\equal{#1}{L}}{\item\item[\textbf{Subscripts}]\item}{}}{%
 \ifthenelse{\equal{#1}{H}}{\item\item[\textbf{Superscripts}]\item}{}}{%
 \ifthenelse{\equal{#1}{W}}{\item\item[\textbf{Abbreviations}]\item}{}}{}}

%\singlespacing
%\renewcommand{\baselinestretch}{1.5}

%\renewcommand{\headrulewidth}{0pt}
%\fancyhead{}

%\setlength{\nomitemsep}{-\parsep}   % removing default extra skip between entries at nomenclature

\numberwithin{equation}{section}    % equation numbers with section numbers
\numberwithin{table}{section}       % table numbers with section numbers
\numberwithin{figure}{section}      % figure numbers with section numbers

% makeindex command needs to run at command prompt to create nomenclature list file
\makenomenclature % makeindex main.nlo -s nomencl.ist -o main.nls
\renewcommand{\nomname}{LIST OF SYMBOLS AND ABBREVIATIONS}

\pagenumbering{arabic}

%\theoremstyle{definition}
%\newtheorem{definition}{Definition}[section]
