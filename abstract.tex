\section*{ABSTRACT}

Lappeenranta University of Technology\\
School of Engineering Science\\
Intelligent Computing Major\\
\\
%\vspace{\stretch{1}}

Maria Glukhova\\
\\
\textbf{Tools for Ensuring Reproducible Builds for Open-Source Software}\\
\\
Master's Thesis\\
\\
2017\\
\pageref{LastPage} pages\\


\begin{tabular}{l p{11.0cm}}  
  
Examiners: & Associate Professor \foreignlanguage{finnish}{Arto Kaarna}\\
& Senior lecturer Vitaly Bragilevskiy\\
Supervisors: & Associate Professor \foreignlanguage{finnish}{Arto Kaarna}\\
& Senior lecturer Vitaly Bragilevskiy\\
& Aidin Hassanzadeh\\

\end {tabular}

Keywords: reproducible builds, open source software, free software, software development, Debian and Linux\\


Reproducible Builds is a collective effort of multiple open-source software
projects, aiming to provide a verifiable path from human readable source code
to the binary code used by computers. To achieve this goal, several tools were
created, allowing for identifying common sources of unreproducibility in build
process.
Diffoscope is one of these tools, which is designed to compare different kinds of
archives, binary formats and other types of files commonly used in software
development process. In this work, several improvements to diffoscope were
designed and implemented, making it more useful for projects working on the
Reproducible Builds.\\
