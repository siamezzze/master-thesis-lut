\section*{ABSTRACT}

Lappeenranta University of Technology\\
School of Engineering Science\\
Intelligent Computing Major\\

%\vspace{\stretch{1}}

Maria Glukhova\\
\\
\textbf{Tools for Ensuring Reproducible Builds for Open-Source Software}\\
\\
Master's Thesis\\
\\
2017\\
\pageref{LastPage} pages, \totalfigures{} figures.\\


\begin{tabular}{l p{11.0cm}}  
  
Examiners: & Associate Professor \foreignlanguage{finnish}{Arto Kaarna}\\
& Senior lecturer Vitaly Bragilevsky\\

\end {tabular}\\

Keywords: reproducible builds, open source software, free software, software development, Debian and Linux\\


The Reproducible Builds project is a collective effort of multiple open-source software
projects, aiming to provide a verifiable path from human readable source code
to the binary code used by computers. To achieve this goal, several tools were
created, allowing for identifying common sources of unreproducibility in build
process.
In this work, an overview of the Reproducible Builds project and the tools designed is made. One of the tools, named diffoscope, is discussed in details; several improvements to this tool are made as part of this work.\\
