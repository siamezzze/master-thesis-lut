\section{Improving usability}

\nomenclature{PYPI}{Python Package Index}
\nomenclature{gzip}{file format and a software application used for file compression and decompression}
\nomenclature{LaTeX}{document preparation system for the \TeX{} typesetting program}
\nomenclature{mtime}{file modification time}
Apart from improving support for specific file types, there are possible improvements
in usability of diffoscope. The idea behind these is to make the tool more user-friendly
and provide output in form that would allow to quickly spot the main differences between
version of package or arbitrary files.

\subsection[Order-like difference]{Order-like difference}
\nopagebreak[4]{
   Sometimes it can be useful to know if texts differ only in lines
   numbering. While these differences should still be reported, as
   they mean the build is not reproducible, there should be a comment
   telling the inputs vary only in line ordering.
}
\subsection[Hiding specific differences]{Hiding specific differences per user
requirements}
\nopagebreak[4]{
    One of the most awaited features of diffoscope is ability to
    hide specific details from output per user requirements.
    That can be useful when there are several possible reproducibility
    issues and user would like to skip information about some of them
    while concentrating on others. One way to do it would be to allow users
    to pass the \texttt{--hide=profiles} option. The most important
    use cases for this feature are as follows \cite{hps}:
    \begin{itemize}[noitemsep]
    \item Ignoring metadata generated by GZIP.
    \item Ignoring all differences in \texttt{control.tar}.
    \item Hiding debug symbols.
    \item Ignoring specific fields in \texttt{.buildinfo} files.
    \item Hiding timestamps generated by \LaTeX.
    \item Ignoring mtimes.
    \item Ignoring build profiles.
    \item Ignoring directories.
    \end{itemize}
    These use cases are documented in \texttt{--hide=profiles} specification
    on Debian wiki \cite{hps}, providing direction for whoever would implement
    the feature.
}

%\cleardoublepage
